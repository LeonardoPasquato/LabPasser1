\documentclass{article}
\usepackage{amsmath}
\usepackage{amsfonts} 
\usepackage{setspace}
\usepackage{anysize}
\usepackage{geometry}
\usepackage{epsfig}
\usepackage{graphicx}
\usepackage{xcolor}
\usepackage{caption}
\usepackage{geometry}
\geometry{a4paper, top=3cm, bottom=3cm, left=2.5cm, right=2.5cm, bindingoffset=5mm}
\captionsetup[table]{position=top, labelformat=empty}
%c'è 1 inch di margine a destra e sinistra
\geometry{margin = 0.9 in}

\title{\huge Report di laboratorio : Reti Logiche}
\author{Leonardo Pasquato, Alessia Rinaldi, Angelo Nutu, Luca Fossa Crescini}
\date{30-11-2022}
\setlength{\parindent}{0cm}

\begin{document}
    \maketitle
    \rule{\linewidth}{0.1mm}

    \section{Obiettivo del laboratorio}

    \section{Componenti}
    \begin{itemize}
        \item Debouncer;
        \item alu;
        \item accumulator;
        \item calcolatrice.
    \end{itemize}
    


    \section{Schema a blocchi}

    \section{Simulazioni}

    \section{Conclusione}

\end{document}