\documentclass{article}
\usepackage{amsmath}
\usepackage{amsfonts} 
\usepackage{setspace}
\usepackage{anysize}
\usepackage{geometry}
\usepackage{epsfig}
\usepackage{graphicx}
\usepackage{xcolor}
\usepackage{caption}
\usepackage{geometry}
\geometry{a4paper, top=3cm, bottom=3cm, left=2.5cm, right=2.5cm, bindingoffset=5mm}
\captionsetup[table]{position=top, labelformat=empty}
%c'è 1 inch di margine a destra e sinistra
\geometry{margin = 0.9 in}

\title{\huge Report di laboratorio : Reti Logiche}
\author{Leonardo Pasquato, Alessia Rinaldi, Angelo Nutu, Luca Fossa Crescini}
\date{30-11-2022}
\setlength{\parindent}{0cm}

\begin{document}
    \maketitle
    \rule{\linewidth}{0.1mm}

    \section{Obiettivo del laboratorio}
    Durante questa esperienza di laboratorio si è studiato e progettata una calcolatrice utilizzando 
    le conoscenze apprese durante il corso.
    Per il progetto è stata utilizzata la scheda FPGA Nexys4, la quale comprende un doppio display
    a 7 segmenti con 4 cifre ciascuno, bottoni e switches (?).\par
    Il progetto della calcolatrice è stato strutturato in modo da poter scegliere attraverso i bottoni quale
    operazione aritmentica effettuare dati due numeri in ingresso (???da dove???) e mostrare sul display
    in codifica $BCD$ il risultato ottenuto. Il calcolo viene effettuato da una Arithmetic Logic Unit (ALU). \par
    Una gestione particolare è stata necessaria nella lettura dei bottoni: un bottone è un componente elettromeccanico, 
    dunque suscetibile al rumore, che sia meccanico o elettromagnetico. Nello specifico, un bottone è composto
    da una lamella conduttrice che apre e chiude un circuito a seconda che il pulsante sia premuto o meno: il rumore può 
    indurre delle attivazioni di breve durata indesiderate. \par
    È stata quindi progettata una entità $Debouncer$ in grado di filtrare tra i segnali ricevuti dai bottoni solamente 
    quelli che possono essere effettivamente riconducibili ad un'azione da parte dell'utente fisico.\par
    È stato gestito inoltre il caso in cui il risultato possa dare un risultato non visualizzabile sul display: è il caso di 
    $overflow$, in cui il risultato risulta essere di una cifra non decodificabile in $BCD$. \par
    È stato quindi gestito l'$overflow$ utilizzando un bit specifico per poter accendere un led sulla scheda 
    Nexys4: nel momento in cui è presente un caso di $overflow$, verrà attivato il corrispondente led. \par

    \section{Componenti}
    \begin{itemize}
        \item Debouncer;
        \item alu;
        \item accumulator;
        \item calcolatrice.
    \end{itemize}
    


    \section{Schema a blocchi}

    \section{Simulazioni}

    \section{Conclusione}

\end{document}